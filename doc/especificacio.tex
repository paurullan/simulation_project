\section{Especificació del problema}

Cal especificar el nostre simulador tant a nivell numèric com en termes de
distribucions estadístiques.

\subsection{Paràmetres d'entrada}

\begin{tabular}{lr}
Paràmetre d'entrada & selecció \\
\toprule
Temps mig de reflexió  & 10 seg \\
Nombre d'usuaris & variable $X$ \\
Velocitat de CPU & 2500 pps \\
Velocitat de disc & 15.000 rpm \\
Peticions de disc per transacció & 5 peticions \\
Temps de resposta del sistema & variable $Y$ \\
Throughput & no \\
Utilització CPU & no \\
Utilització disc & no \\ 
\end{tabular}

\subsection{Modelització dels components}

El nostre simulador està caracteritzat per tres components: la població, la CPU
i el disc dur.

\subsubsection{La població}

Els distints usuaris entren al sistema fent una petició i, un cop resolta,
s'esperen un temps de reflexió.

\paragraph{Distribució de la població}

La població es pot modelar mitjançant una distribució exponencial de mitja 10
segons, $Exp(10)$.

\subsubsection{La CPU}

S'ha escollit un processador comercial AMD Opteron 6140, de capacitat de
treball de 62,4K MIPS\cite{cpu_specs}. Aquest valor és pel conjunt dels vuit
nuclis però podem limitar la simulació a un sol nucli, donant aproximadament
uns 7,5K MIPS. Suposam que una visita són 3M instruccions de promig. La
capacitat de servei i el temps de servei mig són:

\[ \mu = 7,5K MIPS \times \frac{1\text{M instruccions}}{1 \text{seg}} 
                \times \frac{1 \text{transacció}}{3 \text{M instruccions}} \]

\[ \mu = 7,5K MIPS \times \frac{1{M instruccions}}{1 {seg}} 
                \times \frac{1 {transacció}}{3 {M instruccions}} \]

\[ \mu = 2.500 \, \frac{transaccions}{s}\]

\[ s = \frac{1}{\mu} = 0,4 \, ms \]

\paragraph{Distribució de la CPU}

La CPU queda modelada amb una distribució constant $C(0,4)$, en ms.

\subsubsection{El disc dur}

S'ha usat un disc dur de servidor model Seagate Cheetah
15K\cite{disc_specs}. Els discs durs clàssics es composen de tres tasques
seqüencials: búsqueda (\emph{seek}), latència i transfarència. Aquestes tasques
es modelen típicament mitjançant distribucions exponencials, uniformes i
constants respectivament.

Aquest disc dur està especificat com:

\begin{itemize}

  \item Velocitat de rotació 15.000 rpm

  \item Temps de seek del fabricant 3,4 ms

  \item Temps de latència del fabriant 2 ms

  \item Velocitat de transfarència 600 MB/s

\end{itemize}

Suposam un conjunt de fitxers grans, com documents PDF, de tamany 1,5MB. Aquest
fitxers també els suposarem dispersats aleatòriament al llarg del disc però amb
els seus blocs continuus. És a dir, obtenir un fitxer consistirà en:

\[ T = seek + lat + trans \]

\paragraph{Càlcul del \emph{seek}}

Com que al nostre exercici sols feim lectures podem usar el temps de
\emph{seek} mig de lectura especificat pel fabricant de 3,4 ms. Aquest
estadístic es comporta com una distribució exponencial $Exp(3,4)$ en ms.

\paragraph{Càlcul de la latència}

El nostre disc té una velocitat de rotació de 15.000 rpm, per la qual cosa
presenta un temps de volta de 4 ms. Això implica una distribució uniforme $U(0,
4)$ en ms. Cal fixar-se que el fabricant ja havia indicat que la latència mitja
era de 2ms.

\[ 15.000 \text{rpm} \, \frac{1 \text{ min}}{60 \text{ seg}} = 250 \text{rps}
= 4 \text{ ms}\]


\paragraph{Càlcul de la transfarència}

El temps de transfarència ve pel temps de rotació sobre un sector. El fabricant
ens especifica que són 600 MB/s i donat que el nostre fitxer és de 1,5MB tardam
2,5ms de transfarència. Això implica una distribució constant de $C(2,5)$ en
ms.

\[ 600 \, \text{MB} \, \frac{1 \text{ tasca}}{1,5 \text{ MB}}
       = 400 \text{ tasques} = 2,5 \text{ ms}\]

\paragraph{Distribució del disc dur}

D'aquesta manera el disc dur es pot modelar mitjançant una distribució composta
per la del \emph{seek}, la latència i la transfarència juntament amb la forma
del nostre fitxer (en ms):

\[ Exp(3,4) + U(0, 4) + C(2,5) \]
