\newpage

\addtolength{\hoffset}{-2cm}

\section{Codi font}

En aquest apèndix trobarem el codi de l'aplicació, el petit programa per
obtenir les gràfiques del transitori i els jocs de prova.

\subsection{Codi general}

Aquí trobarem els distints fitxers del la nostra pràctica:

\begin{description}

  \item[main.py] Programa principal.

  \item[suite.py] Gestor de tasques i llançament de rèpliques.

  \item[sim.py] El mòdul principal de la simulació.

  \item[model.py] Modelització dels usuaris, la cpu i el disc.

  \item[rand.py] El nostre mòdul de nombres aleatoris.

  \item[statistic.py] Mòdul del càlcul estadístic.

  \item[chart\_generator.py] Mòdul per pintar les gràfiques.

\end{description}

\pycode{main}
\pycode{suite}
\pycode{sim}
\pycode{model}

\pycode{rand}
\pycode{statistic}
\pycode{chart-generator}

\newpage

\subsection{Càlcul del transitori}

Hem usat aquest petit programa per obtenir un parell de mostres gràfiques i
així tenir pistes sobre com determinar el transitori.

\pycode{calcul-transitori}

\newpage

\subsection{Jocs de prova}

Per la tasca de programació hem aprofitat per aplicar per primera vegada un
model de programació anomenat \emph{programació dirigida per proves} (de
l'anglès \emph{test driven development} o TDD). El que es fa és \emph{primer}
escriure els jocs de proves per cadasqun dels mètodes que es van necessitant i
no s'avança a la següent característica fins que totes les proves anteriors
funcionen. Aquesta metodologia pot pareixer més engorrosa però un cop el
programa va més enllà d'un petit mòdul i alguna gràfica s'accelera la
programació per que és molt senzill detectar els problemes causats pel nou
desenvolupament.

Aquest comentari el feim per indicar que el gran nombre de tests de a causa de
l'estil de programació poc habitual.

\pycode{test-chart}
\pycode{test-rand}
\pycode{test-statistic}
\pycode{test-model}
\pycode{test-sim}
\pycode{test-sim-queue}
\pycode{test-suite}

\newpage

\addtolength{\hoffset}{+2cm}
